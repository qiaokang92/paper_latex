\chapter{总结,结论与下一步的工作}

快速总结如下,绝对保护可以通过绝对的孤立主义很容易地获得,但这通常是不可接受的解决方案。其他形式的保护似乎都取决于使用极其复杂和/或资源密集型的分析技术,不精确的解决方案往往使系统可以正常使用的时间更少。


预防似乎涉及了限制合法的活动,而治疗在没有拒绝服务的条件下可能非常困难。精确的检测是不可判定的,然而统计方法可以在时间或程度上用来限制未被发现的传播。典型的使用行为是必须充分理解,以便于使用统计方法,而且这种行为是因系统而异的。有限形式的检测和预防可以用于对病毒提供有限的保护。

已经表明,病毒有可能通过特定的允许共享的系统进行扩散。目前每一个使用的通用系统可以至少限制病毒的攻击。在许多当前“安全”系统中,由不受信任的用户创建时,病毒倾向于进一步蔓延。实验表明了病毒攻击的可行性,病毒传播的迅速性以及很容易在各种操作系统上创建的特性。进一步的实验仍在进行之中。

给出的结果不是操作系统也不是实现相关的,但都是基于系统的基本性质。更重要的是,它们反映了系统目前使用的假设。此外,几乎每一个正在发展的“安全”系统目前都基于Bell-LaPadula或晶格策略,而这个工作已经清楚地表明这些模型不足以防止病毒的攻击。病毒基本上证明了完整性控制必须作为任何安全操作系统的一个重要组成部分。


对病毒和对策,一些不可判定的问题已经确定了。对一些潜在的对策有一定深度的研究,但是也没有提供理想的解决方案。本文提出的几个技术可以提供有限的病毒防护措施,在目前也只有有限的作用。完全安全地抵御病毒的攻击,系统必须防止传入的信息流,而对泄漏信息,系统必须安全防范即将离开的信息流。为了让系统允许共享,必须有一些信息流动。因此本文的主要结论是,在一个通用的多级安全系统中共享的目标可能与在病毒安全防护的目标相反,这使得让它们和解并共存是不可能的。


最重要的正在进行的研究涉及到对计算机网络上病毒影响的研究。主要感兴趣的是确定病毒如何快速蔓延到世界上大部分的计算机。这是通过简化数学模型和对“典型的”计算机网络病毒传播的研究来进行的。对在安全网络上的病毒的影响也有极大的兴趣。病毒让我们相信,一个系统的完整性和安全性必须得到保证以防止病毒攻击,网络也必须保持这两个标准以保障多级计算机之间的共享。这就引入了在这些网络上重要的限制。


演化程序的显著例子已经在资源水平上开发生产了许多给定的程序的演化。一个简单的进化病毒和一个简单的进化抗体也正在研发之中。
