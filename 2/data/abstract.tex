% !Mode:: "TeX:UTF-8"
\begin{cabstract}
%%%

本文研究如何在VxWorks系统镜像
文件中插入代码,
并通过修改指令流劫持特定的函数,
从而在该函数被调用时执行插入代码。
二进制插入和劫持操作可为VxWorks平台下的
系统测试与调试、攻击防范等提供研究基础。

%%代码插入意味着通过将二进制机器码注入到可执行文件中,
%%使这段代码随着程序执行被映射到内存空间。
%%劫持则是指在程序合适的位置插入跳转指令,
%%从而在特定的时机获取控制权,使插入代码被执行。
代码插入是实施劫持的前提;
也只有通过合适的劫持操作,
插入代码才能得到运行。
针对Linux等平台的ELF文件,
存在若干成熟的插入技术。
然而由于VxWorks的若干特性,
确定唯一适合VxWorks系统的代码插入方案为
使用nop指令串进行插入。
为了方便大规模插入代码,编写了自动化工具。

程序劫持技术中,主要有修改入口点
和修改函数头两种方法。
分析认为修改入口点不适用于对VxWorks的劫持,
而修改函数头的方法只能劫持函数入口。
本文提出了一种可以劫持函数体任意位置的方法。
即通过一段中间代码来间接调用插入代码,
并在中间代码中进行寄存器的保存和恢复工作。
这种方法不但对劫持入口没有限制,
而且有效地分离了劫持操作与劫持函数的编写。

利用上述的插入和劫持方案,
实现了一个实例mini-notify。
通过对VxWorks二进制镜像进行插入和劫持修改,
可以实现类似Linux内核中
inotify的文件系统监控功能。
实验证实包含mini-notify的系统可以
截获所有的针对本地文件系统的文件操作。

\end{cabstract}
\begin{eabstract}
The paper aims at injecting codes to VxWorks image file 
and hijacking given functions of VxWorks by modifying the 
instructions. 
Injecting and hijacking can be used in closed-source software 
debugging, intrusion detection and intrusion prevention.
 
The final goal is to hijack instruction stream
 which is based on injecting codes.
Many present ways to inject code in ELF on Linux 
can be used potentially.
However, only one way, injecting code into nops, 
is available as a result of the features of VxWorks. 
A tool is written to help inject.
 
As for hijacking, exsiting ways usually modify
the entrypoint of ELF or the starting 
instructions of a function.
To modify the entrypoint doesn't apply to VxWorks image.
On the other hand, to modify the 
starting instructions means the hijack point 
cannot be anything but the entrance of the function.
A new method is proposed which hijack functions 
using some medial codes, which calls the 
injected code and help store and restore 
the registers. 
This new method helps hijack any position inside a 
function and write more powerful injected code.

Finally, an example, in which we hijacked dosFs file system 
in VxWorks by injecting code and hijacking some functios, is given.
Each file operation such read and write will be 
detected and reported.   





\end{eabstract}
